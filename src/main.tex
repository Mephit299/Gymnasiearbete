\documentclass[11p]{article}
% Packages
\usepackage{amsmath}
\usepackage{graphicx}
\usepackage{fancyheadings}
\usepackage[swedish, english]{babel}
\usepackage[
    backend=biber,
    style=authoryear-ibid,
    sorting=ynt
]{biblatex}
\usepackage[utf8]{inputenc}
\usepackage[T1]{fontenc}
\usepackage{titlesec}
\usepackage{hyperref}
%Källor
\addbibresource{references.bib}
\graphicspath{ {./images/} }

% Lite variabler
\def\email{levi.hogdal@elev.ga.ntig.se}
\def\foottitle{Gymnasiearbete} % Kanske borde ändras
\def\name{Levi Högdal}

\title{Gymnasiearbete \\ \small Gymnasiearbete}
\author{\name}
\date{\today}


\begin{document}



% fixar sidfot
    \lfoot{\footnotesize{\name \\ \email}}
    \rfoot{\footnotesize{\today}}
    \lhead{\sc\footnotesize\foottitle}
    \rhead{\nouppercase{\sc\footnotesize\leftmark}}
    \pagestyle{fancy}
    \renewcommand{\headrulewidth}{0.2pt}
    \renewcommand{\footrulewidth}{0.2pt}

% i Sverige har vi normalt inget indrag vid nytt stycke
    \setlength{\parindent}{0pt}
% men däremot lite mellanrum
    \setlength{\parskip}{10pt}
    \begin{otherlanguage}{swedish}


        \begin{titlepage}
            \centering

            % Title and subtitle are enclosed between two rules.
            \rule{\textwidth}{1pt}

            % Title
            \vspace{.7\baselineskip}
            {\huge \textbf{Tillgänglighet på Sveriges kommuner}}

            % Subtitle
            \vspace*{.5cm}
            {\LARGE Kommunernas anpassning till WCAG 2.2 standarden och generellt användarvänlighet}

            \rule{\textwidth}{1pt}

            \vspace{1cm}

            % Set this size for the remaining titlepage.
            \large

            % Authors side by side, using two minipages as a trick.
            \begin{minipage}{.5\textwidth}
                \centering
                \name \\
                {\normalsize \url{levihogdal@elev.ga.ntig.se}}
            \end{minipage}%

            \vspace{3cm}

            % Report logo.
            \includegraphics[width=0.4\textwidth]{../images/NTI Gymnasiet_Symbol_print_svart.png}

            \vfill

            % University and date information at the bottom of the titlepage.
            NTI Gymnasiet Umeå \\
            Teknikprogrammet\\
            Gymnasiearbete\\
            Datum: \today \\
            Handledare: Jens Andreasson
        \end{titlepage}
    %\maketitle
    %\begin{center}
    %\includegraphics[width=0.4\textwidth]{../images/NTI Gymnasiet_Symbol_print_svart.png}
    %\end{center}
    \end{otherlanguage}


    \newpage
    \begin{otherlanguage}{english}
    \begin{abstract}
        Accessibility on the internet is important for the user and the creator of a website.
        All users should be abel to use a website even if they have a disability.
        This study aims to test the accessibility guidelines WCAG 2.2 on the websites of swedish townships (kommuner).
        Because townships in general were tested the study also includes if townships with less population meet less of the WCAG 2.2 criteria.
        The method to accomplish this was simply test each criteria of WCAG 2.2 on all the chosen websites.
        How each criteria of WCAG 2.2 was tested and the result were combined in a spreadsheet that can be found at the end of the study.
        None of the websites meet all the tested WCAG 2.2 criteria.
        The worst performing website meet 47/70 of the tested criteria and the best performing website met 63/70 of the tested criteria.
        The study also showed that there might be a tendency for townships with less population to have a worse performing website but to draw such a conclusion more websites wold have to be tested.

    \end{abstract}
    \end{otherlanguage}
    \begin{otherlanguage}{swedish}
    \newpage
    \tableofcontents
    \newpage

    \section{Inledning}
    Att kunna använda en hemsida på ett bra sätt är något många nuförtiden tar för givet.
    Att förstå internet och kunna enkelt läsa och navigera olika forum och hemsidor är inte något som personer som har använt internet har problem med.
    Det finns många olika sätt som en hemsida kan se utt men hur ska hemsidor anpassas för folk som har svårt att använda internet.
    Om användaren har någon funktionsnedsättning så kan hemsidan bli väldigt svår att använda och tolka om den inte är anpassad för deras användning.
    För att underlätta utvecklingen av användarvänliga hemsidor har en organisation som heter W3C skapat standarder(WCAG) för användarvänlighet och tillgänglighet.

    
    \subsection{Syfte och frågeställning}
    Hur mycket hemsidor förhåller sig till WCAG standarden är varierande men jag är intresserad av hur olika kommunala hemsidor förhåller sig till dem och vilka olika delar av WCAG 2.2 standarden de möter.
    \\Det jag vill undersöka är:
    \begin{itemize}
        \item Vilka WCAG 2.2 krav uppfyller kommuners hemsidor och vilka skillnader finns det i kraven som kommunerna uppfyller.
        \item Finns det några skillnader på antalet invånare i kommunen och hur många krav kommunen uppfyller.
    \end{itemize}

    \section{Bakgrund}

    \subsection{Tillgänglighet}
    Tillgänglighet i det här sammanhanget betyder inte hur åtkomligt något ska vara utom att hemsidan ska vara användbar för alla användare, även de med funktionsnedsättning.
    Tillgänglighet handlar om hur enkelt användaren kan fösta hemsidan \parencite{webbriktlinjer}.
    Det innehåller till exempel :
    Hur bra kontrast det är mellan bakgrund och text så att det ska vara enkelt att läsa texten.
    Har bilder en bildtext?
    Blinkar hemsidan för mycket?
    \\Inte kan jag komma åt hemsidan.
    \\Webbsidor design och utformning ska vara anpassad så att alla människor ska kunna använda och förstå hemsidan så länge de kan läsa språket.
    För att hemsidor ska kunna användas av folk med funktionsnedsättningar så görs standarder som man kan följa när man skapar hemsidor och olika lagar.
    Webbstandarden som den här undersökningen utgår från är WCAG som är utgivet av W3C.


    \subsection{Hemsidors uppyggnad}
    Hemsidor brukar vara uppbyggda i grunden av 4 olika delar:
    \begin{itemize}
        \item Hypertext markup language (HTML)
        \item Cascading style sheets (CSS)
        \item Accessible rich internet applications (ARIA)
        \item Javascript/Typescript (JS/TS)
    \end{itemize}
    Den här undersökningen är intresserad i HTML, CSS och ARIA men inte Javascript/Typescript.
    Javascript och Typescript är inte relevant till den här undersökningen.

    \subsubsection{Hypertext markup language}
    % Du har feedback. Skriv om. %
    Hypertext markup language (HTML) är standardspråket som hemsidor är gjorda av.
    HTML är språket som ger struktur och definierar grunden till hur hemsidan ser ut \parencite{HTML}.
    Ett element är skriven på följande vis <body> och </body>.
    Text skriven utanför ett element förlorar all semantisk mening och kommer att skrivas ut på hemsidan som normal text.
    Om den istället skrivs mellan början och slutet av ett element kommer att bli påverkad av elementets funktioner och stilar.
    Det gör att texten automatiskt formateras efter elementet och program kan förstå innehållet.

    %Text som inte är i en tagg kommer att hanteras som normalt textinnehåll och skrivas upp på hemsidan som normal text.
    %Om texten skrivs mellan en öppnings tag och stängnings tagg kommer texten bli påverkan av elementets funktioner och stilar.
    %Den första <body> kallas för en öppnings tagg och är början av elementet och den andra </body> kallas för stängnings tagg och är slutet av elementet.
    %Man kan också skriva mer i en tagg till exempel <article id="article"> där det första ordet är vilket element det är och den andra i det här fallet är ett id som kan användas för att hitta och interagera elementet med exempelvis javascript.
    %En stängnings tag är alltid på följande vis: </p>, </body>, </main>, </head>, </div> och så vidare.

    \subsubsection{Cascading style sheets}
    Cascading style sheets (CSS) är ett språk som används för att bestämma hur en hemsida ser ut \parencite{CSS}.
    Grundstilarna i HTML gör en hemsida som fungerar men den kommer inte nödvändigtvis se ut som utgivaren av sidan vill.
    För att justera hur en hemsida ser ut och bestämma des design används CSS.

    \subsubsection{Accessible rich internet applications}
    Accessible rich internet applications (ARIA) är extra roller och attribut som används för att göra en hemsida mer tillgänglig.
    ARIA har mest funktionalitet om man inte navigerar hemsidan som en normal användare utan med andra hjälpmedel \parencite{ARIA}.
    Att implementera ARIA på ett felaktigt sätt gör en hemsida mindre användarvänlig och om det går bör man använda HTML element och attribut om det uppfyller samma funktion.
    
    \subsection{The world wide web consortium}
    The world wide web consortium (W3C) är det företaget som är utgivarna WCAG 2.2 standarden \parencite{W3C}.
    W3C är ett företag som grundades av Tim Berners-Lee med syftet att säkerställa webbens utveckling och framtid.
    För att uppnå målet har de skapa standarder för webben och har diskussioner med flertal grupper och organisationer om webben.
    WCAG standarden har de skapat och det är den som undersökningen kommer att utgå ifrån.

    \subsection{Web content accessibility guidelines}
    Web content accessibility guidelines (WCAG) är en lista av riktlinjer för webben \parencite{WCAG_2.2}.
    Målet med riktlinjerna är att göra webben mer tillgänglig och användarvänlig så att alla användare kan använda hemsidan.
    En del av WCAG standarden är att alla individuella krav ska kunna testas separat.
    WCAG 2.2 är den senaste versionen av WCAG när undersökningen gjordes och den är indelad i 4 stora kategorier.
    \\De olika delarna är:
    \begin{enumerate}
        \item Märkbart
        \item Manöverbart
        \item Begriplighet
        \item Robust
    \end{enumerate}

    Märkbart handlar om att information och innehåll ska vara synligt och tillgängligt för användare.
    Ljudbaserad media ska vara tillgängligt för användare som är döva.
    Den visuella presentationen av hemsidan ska vara läsbar och enkel att förstå.

    Manöverbart handlar om att hemsidan ska vara navigerbar på ett användarvänligt vis.
    Det finns flera vägar än bara mus och tangentbord att navigera på en hemsida och de behöver också fungera.
    Hemsidor ska till exempel vara navigerbar med pekskärmar och endast tangentbord.

    Begriplighet handlar om att användaren ska förstå innehållet på hemsidan.
    Att förstå innehållet på en hemsida är inte alltid det lättaste och det är ännu svårare om användaren har en funktionsnedsättning.
    Navigationen och språket på hemsidan ska vara tydligt och eventuella fel som användaren gör ska förklaras.

    Robust relaterar till att olika program ska kunna bestämma vad innehåll betyder och gör.
    Semantisk mening i de olika elementen är viktiga för att program ska kunna tolka hemsidan på ett bra sätt.

    \subsubsection{Varför WCAG 2.2}
    WCAG är globalt sett den accepterade webbstandarden och om man kollar på lagar och regler som finns i sverige och EU ser man att det utgår från WCAG.
    I den svenska myndighetens kriterier för hur webbplatser ska användarvänlighets anpassas finns det direkt länkat till vilket WCAG som kravet utgår ifrån. \parencite{Utförande_av_Dos_lagen}


    \subsection{Lagar}
    I Sverige finns en lag som heter "Lagen om tillgänglighet till digital offentlig service (DOS-lagen)"\parencite{Dos-lagen}.
    Lagen ställer krav på offentliga aktörer så att de ska tillgänglighetsanpassa webbplatser och mobila applikationer.
    Myndigheten för digital förvaltning (Digg) är ansvarig för att genomföra lagen och de utgår från WCAG standarden för att sätta krav på hemsidorna \parencite{Utförande_av_Dos_lagen}.

    \subsubsection{Offentlig aktör}
    Hur en offentlig aktör definieras finns förklarat i Dos-lagen \parencite{Dos-lagen} och av DIGG \parencite{Om_Dos-lagen} men det kan enklare förklaras som offentlig information från staten som statliga och kommunala myndigheter samt sammanslutningar till dem.
    Information som handlar om utbildning, skola, sjukvård ock omsorg måste också följa lagen.
    Även om det ägs privat så kommer de behöva följa kraven från Dos-lagen.

    \subsection{Digitala verktyg}
    För att underlätta undersökningen användes 2 digitala verktyg.
    Det första heter web accessibility evaluation tools (WAVE).
    WAVE är ett digitalt verktyg som tillåter automatiskt testning för några WCAG krav \parencite{WAVE}.
    WAVE kollar kontrasten på innehållet, redovisar HTML strukturen och vart ARIA används på hemsidan.
    Fell i HTML syntax redovisas enkelt och ikoner på hemsidan visas var olika element och attribut används.

    \begin{center}

    \includegraphics[width=0.4\textwidth]{../images/WAVE.png}

    WAVE resultatet från Sorsele kommuns startsida.
    \end{center}

    \subsubsection{Screen reader}

    En screen reader är ett digitalt verktyg som läser upp ett html element och dess textinnehåll.
    Det finns WCAG krav där användaren programmässigt ska kunna bestämma vad något är och vad det har för funktion.
    Till exempel finns WCAG krav 1.3.5 Identify Input Purpose (Level AA) där användaren ska kunna bestämma vilken information ett textinmatningsfält ska få in.
    Med en screen reader ska den berätta för användaren att det är ett textinmatningsfält och vilken information som ska skrivas in.

    \subsection{Kommunala Hemsidor}
    För att meddela information om en kommun digitalt har kommuner hemsidor som det styr innehållet på och lägger upp information om kommunen.
    Det som är intressant om kommunal hemsidor är att alla hemsidor inte ser likadan ut utan de flesta ser annorlunda.
    Det finns också några kommuner med hemsidor som sek likadana ut bara att de har varierande innehåll.
    %Vissa kommuner har hemsidor som ser likadana ut bara varierande innehåll.
    Skillnaden i hur kommunala hemsidor ser ut betyder att det finns en skillnad på hur de är uppbyggda och hur det är anpassade för användarvänlighet för webben.
    
    Hemsidorna i undersökningen valdes från listan på \textcite{SverigesKommuner}.
    Listan är ordnad i folkmängd, störst till minst för alla kommuner i Sverige.
    Till undersökningen valdes en hemsida som var bland de högst upp på listan, en i mitten av listen och en längst ner på listan.
    Det valda kommunerna blev \textcite{Linköpings_kommun} (störst), \textcite{Höörs_kommun} (i mitten) och \textcite{Sorsele_kommun} (minst).
    
    \section{Metod och Material}

    Undersökningen utgick på att testa hur kommunernas hemsidor uppfyller alla krav i WCAG 2.2.
    Digg och W3C har givit ut en listor över hur kraven kan uppnås i lagen och WCAG 2.2.
    För specifikation om WCAG kraven se \textcite{WCAG_2.2}
    Undersökningen utgick från en sammanställning över WCAG 2.2 kraven och hur de kan mötas.
    Den sammanställningen och resultatet finns i ett kalkylark (se bilaga 1).
    Utifrån tabellen testades alla kraven på hemsidorna.
    Nedan finns en förklaring av vad varje colum i kalkylarket består av.
    \begin{enumerate}
        \item Själva WCAG 2.2 kravet som skulle testas.
        \item Hur WCAG 2.2 kravet ska testas
        \item På vilken hemsida som kravet testades
        \item Om hemsidan uppfyllde kravet
        \item Kommentarer om hemsidan
        \item På vilken hemsida som kravet testades
        \item Om hemsidan uppfyllde kravet
        \item Kommentarer om hemsidan
        \item På vilken hemsida som kravet testades
        \item Om hemsidan uppfyllde kravet
        \item Kommentarer om hemsidan
    \end{enumerate}

    \section{Resultat}

    På alla hemsidor så testades inte 16 av de 86 kraven.

    Det finns 3 krav som inte testades för att jag inte hade en bra väg att kunna testa dem.
    Krav 2.2.5 Re-authenticating (Level AAA) vilket handlar om att användaren loggar in igen efter att den har blivit utloggad utan dataförluster.
   \\ Krav 3.3.4 Error Prevention(Legal, Financial, Data) (Level AA) vilket handlar om förebyggande av fel vid finansiella transaktioner och lagligt bindande kontrakt.
    Till exempel att informationen ska kollas efter fel när användaren skriver in information för att köpa något.
  \\  Krav 3.3.6 Error Prevention (All) (Level AAA) vilket är typ samma som 3.3.4 fast det är inte endast för finansiella transaktioner och kontrakt.
    All information som användaren skickar in ska kollas efter fel och användaren ska kunna åtgärda de felen.

    11 av kraven inom den första kategorin av märkbart testades inte.
    Kraven handlade om ljud och innehåller saker som att det ska finnas undertexter och teckenspråkstolkning för videor och ljud med mera.
    De testades inte av den enkla anledningen av att jag inte hittade någon ljudfil eller videofil på hemsidorna.

    Krav 2.1.4 Character Key Shortcuts (Level A) handlar om tangentbordsgenvägar och testades inte för jag är inte medveten om några tangentbordsgenvägar på hemsidorna.

    Krav 2.2.4 Interruptions (Level AAA) handlar om att sidan uppdateras i nutid med ny information.
    Det kravet testades inte för hemsidan hade inga delar som uppdaterades i nutid.

  \\  I slutändan testades 70 av de 86 kraven på hemsidorna.
    %Kanske det lättaste är att bara ha en tabbel och sen utgå från den tabbelen.%
    \subsection{Märkbart krav}
    I den första delen av WCAG märkbart finns 29 krav var 18 av de testades.
    Sorsele kommun mötte 10 och misslyckades med 8 av dem.
    Totalt lyckades kommunen med 55.56$\%$ av märkbart kraven.
    \\Höörs kommun mötte 13 och misslyckades med 5 av dem.
    Totalt lyckades kommunen med 72.22$\%$ av märkbart kraven.
    \\Linköpings kommun mötte 14 och misslyckades med 4 av dem.
    Totalt lyckades kommunen med 77.78$\%$ av märkbart kraven.
    \\I den här kategorin ligger Sorsele kommun klart lite bakom de andra kommunerna medans Linköping och Höörs kommun skiljer det endast 1 krav.

    \begin{center}
    Märkbart (18 av 29 krav testades)

    \begin{tabular}{ |c|c|c|c|}
        \hline
        Kommun & Möter & Möter inte & Möter ($\%$) \\  \hline
        Sorsele & 10 & 8 & 55.64$\%$ \\ \hline
        Höör & 13 & 5 & 72.22$\%$ \\ \hline
        Linköping & 14 & 4 & 77.78$\%$ \\ \hline
    \end{tabular}
    \end{center}

    \subsection{Manöverbart krav}
    I den andra delen av WCAG manöverbart finns 34 krav var 31 av de testades.
    Sorsele kommun mötte 21 och misslyckades med 10 av dem.
    Totalt lyckades kommunen med 67.74$\%$ av manöverbar kraven.
    \\Höörs kommun mötte 30 och misslyckades med 1 av dem.
    Totalt lyckades kommunen med 96.77$\%$ av manöverbar kraven.
    \\Linköpings kommun mötte 30 och misslyckades med 1 av dem.
    Totalt lyckades kommunen med 96.77$\%$ av manöverbar kraven.
    \\Även i den här kategorin ligger Sorsele bakom med Linköping och Höörs kommun möte lika många krav.

    \begin{center}
    Manöverbar (31 av 34 krav testades)

    \begin{tabular}{ |c|c|c|c|}
        \hline
        Kommun & Möter & Möter inte & Möter ($\%$) \\  \hline
        Sorsele & 21 & 10 & 67.74$\%$ \\ \hline
        Höör & 30 & 1 & 96.77$\%$ \\ \hline
        Linköping & 30 & 1 & 96.77$\%$ \\ \hline
    \end{tabular}
    \end{center}

    \subsection{Begriplighet krav}
    I den tredje delen av WCAG begriplighet finns 21 krav var 19 av de testades.
    Sorsele kommun mötte 16 och misslyckades med 3 av dem.
    Totalt lyckades kommunen med 84.21$\%$ av begriplighets kraven.
    \\Höörs kommun mötte 17 och misslyckades med 2 av dem.
    Totalt lyckades kommunen med 89.47$\%$ av begriplighets kraven.
    \\Linköpings kommun mötte 18 och misslyckades med 1 av dem.
    Totalt lyckades kommunen med 94.73$\%$ av begriplighets kraven.
    \\Fortsatt är Linköpings kommun högst i antal möta krav med Höörs kommun rakt bakom och Sorsele kommun i sista plats

    \begin{center}
    Begriplighet (19 av 21 krav testades)

    \begin{tabular}{ |c|c|c|c|}
        \hline
        Kommun & Möter & Möter inte & Möter ($\%$) \\  \hline
        Sorsele & 16 & 3 & 84.21$\%$ \\ \hline
        Höör & 17 & 2 & 89.47$\%$ \\ \hline
        Linköping & 18 & 1 & 94.73$\%$ \\ \hline
    \end{tabular}
    \end{center}

    \subsection{Robust krav}
    Den fjärde och sista delen av WCAG är robust och där finns det bara 2 krav där båda testades.
    Sorsele kommun mötte 0 krav.
    Höörs kommun mötte 1 krav
    Linköpings kommun mötte 1 krav.

    \begin{center}
    Robust (2 av 2 krav testades)

    \begin{tabular}{ |c|c|c|c|}
        \hline
        Kommun & Möter & Möter inte & Möter ($\%$) \\  \hline
        Sorsele & 0 & 2 & 0$\%$ \\ \hline
        Höör & 1 & 1 & 50$\%$ \\ \hline
        Linköping & 1 & 1 & 50$\%$ \\ \hline
    \end{tabular}
    \end{center}
    
    \subsection{Alla krav}
    Totalt lyckades Sorsele kommun möta 47 av kraven och misslyckades med 23 av dem vilket betyder att de mötte totalt 67.14$\%$ av kraven.
    Höörs kommun lyckades möta 61 av kraven och misslyckades med 9 av dem vilket betyder att de mötte total 87.14$\%$ av dem.
    Linköpings kommun lyckades möta 63 av de 70 kraven och mötte 90$\%$ av alla testade krav.

    \begin{center}
    Totalt (70 av 86 krav testades)

    \begin{tabular}{ |c|c|c|c|}
        \hline
        Kommun & Möter & Möter inte & Möter $\%$ \\  \hline
        Sorsele & 47 & 23 & 67.14$\%$ \\ \hline
        Höör & 61 & 9 & 87.14$\%$ \\ \hline
        Linköping & 63 & 7 & 90$\%$ \\ \hline
    \end{tabular}
    \end{center}

    \section{Diskussion}
    Resultatet av Linköping kommun och Höörs kommun är väldigt lika.
    Det skiljer endast två krav mellan deras resultat.
    Linköping möter tre krav som Höörs inte gör och Höörs ett krav som Linköpings inte möter.
    Det är inte en så stor skillnad i antal möta krav som kanske skulle se likadan ut om andra kommuner valdes.
    Jag tycker att det inte kan dras någon slutsats från den skillnaden utan jag skulle behöva testa mer hemsidor för att dra en slutsats.

    Den riktigt intressanta delen är Sorsele kommun som möter 14 mindre krav än Höörs kommun.
    Det är en drastisk skillnad jämfört med Höör och Linköping.
    Sorsele kommun klarar mindre krav i alla kategorier men det är mest i manöverbar kategorin.
    Väldigt många av de misslyckande kraven relaterar till att användaren kan använda tab knappen på tangentbordet för att komma åt ett sökfält som inte visas.
    På Sorseles hemsida fanns det vissa funktioner bara på vissa delar av hemsidan.
    Om man tryckte på länkar för att gå vidare försvann de funktionerna och hur hemsidan såg ut ändrades.
    Att navigera från till exempel "Omsorg och stöd" till "Orosanmälan för barn eller vuxen" ändrar helt sidans layout.
    Det var inte något som jag märkte på Linköpings eller Höörs kommun hemsida.

    %Det här är väldigt subjektivt men känslan jag fick av Sorsele kommun var att den blev skapad och sen hade blivit uppdaterad ett flertal gånger av olika personer.
    %Att skapa en hemsida och sen uppdatera den flera gånger av olika företag med nya funktioner och efter nya regler kostar förmodligen mindre än att skapa en ny hemsida men hemsidan kommer förmodligen att behålla några av de dåliga delarna före uppdateringen.
    %Det är en kommun med liten befolkning så att uppdatera en sida är förmodligen mer logiskt med tanken på kommunens lägre inkomster än resten.
    %Jag fick inte den känslan av Linköpings eller Höörs kommun hemsida.

    För att resultatet ska vissa något om kommuner generellt skulle fler kommuner undersökas.
    Den enda slutsatsen jag kan dra är att det kanske finns en tendens till att kommuner med lägre antal invånare har en mindre användarvänlig hemsida.
    Att undersöka fler kommuner är också en väg att förbättra undersökningen och förstärka resultatet.
    Att testa fler kommuner på endast en del av WCAG 2.2 skulle förmodligen ge ett resultat där fler slutsatser skulle kunna dras.
    %Det går också att fördjupa sig mer i individuella WCAG krav och kontrollera dem mer noggrant.
    %\textcite{WCAG_2.2} innehåller för varje krav länkar med extra förklaring av innebörden av kravet och förslag på olika väga att testa krav.

    \newpage
    \section{Referenser}

    \printbibliography[heading=none]

    \section{Bilagor}
    Bilaga 1
    Google spreadsheet:
    \\ https://docs.google.com/spreadsheets/d/1SNX30NQTKxSrTisLqjCoPWqg7aKbZACcXsoCPYrQb6c/
    edit?usp=sharing

    \end{otherlanguage}
\end{document}
