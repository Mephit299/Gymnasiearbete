\documentclass[11p]{article}
% Packages
\usepackage{amsmath}
\usepackage{graphicx}
\usepackage{fancyheadings}
\usepackage[swedish]{babel}
\usepackage[
    backend=biber,
    style=authoryear-ibid,
    sorting=ynt
]{biblatex}
\usepackage[utf8]{inputenc}
\usepackage[T1]{fontenc}
\usepackage{hyperref}
%Källor
\addbibresource{references.bib}
\graphicspath{ {./images/} }

% Lite variabler
\def\email{levi.hogdal@ga.ntig.se}
\def\foottitle{PMmall}
\def\name{Levi Högdal}

\title{PMmall \\ \small Gymnasiearbete}
\author{\name}
\date{\today}

\begin{document}

% fixar sidfot
    \lfoot{\footnotesize{\name \\ \email}}
    \rfoot{\footnotesize{\today}}
    \lhead{\sc\footnotesize\foottitle}
    \rhead{\nouppercase{\sc\footnotesize\leftmark}}
    \pagestyle{fancy}
    \renewcommand{\headrulewidth}{0.2pt}
    \renewcommand{\footrulewidth}{0.2pt}

% i Sverige har vi normalt inget indrag vid nytt stycke
    \setlength{\parindent}{0pt}
% men däremot lite mellanrum
    \setlength{\parskip}{10pt}

    \maketitle

    \newpage

    \section{Inledning}
    Att kunna använda en hemsida på ett bra sätt är något många nuförtiden tar för givet.
    Att förstå internet och kunna enkelt läsa och navigera olika forum och hemsidor är inte något som personer som har använt internet.
    Det finns många olika sätt som en hemsida ska se utt och designas men hur ska hemsidor anpassas för folk som har svårt att använda internet.
    Om man har någon funktionsnedsättning eller om det är väldigt dålig kontrast på en hemsida så blir den svår att använda.
    För att underlätta så att allt inte är kaos och användaren ska kunna använda hemsidan på ett bra sätt har en organisation som heter W3C skapat standarder(WCAG) för användbarhet på hemsidor.
    Hur mycket hemsidor förhåller sig till WCAG standarden är varierande men jag är intresserad i hur olika kommunala hemsidor förhåller sig till dem.
    Det jag vill undersöka är:
    Vilka WCAG2.2 krav uppfyller kommuners hemsidor och vilka skillnader finns det i kraven som kommunerna uppfyller.

    \section{Bakgrund}

    \subsection{Användarbarhet}

    \subsection{Tillgänglighet}
    Tillgänglighet i det här sammanhanget betyder inte hur åtkomligt något ska vara utom att saker ska kunna användas av alla personer även de med funktionsnedsättning.
    Tillgänglighet handlar om saker som hur bra kontrast det är mellan bakgrund och text så att det ska vara enkelt att läsa texten och har bilder en bildtext, inte kan jag komma åt hemsidan \parencite{webbriktlinjer}.
    Webbsidor design och utformning ska vara anpassad så att alla människor ska kunna använda och förstå hemsidan så länge de kan läsa språket.
    För att hemsidor ska se vettiga ut och användas av folk med funktionsnedsättningar så görs standarder som man kan följa när man skapar hemsidor och olika lagar.
    Webbstandarden som den här undersökningen kommer hålla sig till är WCAG.

    \subsection{WCAG 2.2}
    WCAG är en webbstandard som är skapad av W3C.
    Enligt \textcite{W3C} är W3C ett företag som grundades 1994 av Tim Berners-Lee för att säkerställa webbens utveckling och framtid.
    För att åstadkomma det målet släpper W3C olika standarder och riktlinjer för webben.
    WCAG 2.0 är en samling av olika rekommendationer som alla ska kunna testas separat och inte riktas mot någon specific teknologi utan webben skännerält. \parencite{WCAG_2.0}



    Riktlinjerna ska kunna testas på en hemsida oberoende hur hemsidan är uppbyggd.

    \subsubsection{Varför WCAG 2.2}

    \subsection{Lagar}
    I sverige finns en lag som heter "Lagen om tillgänglighet till digital offentlig service (DOS-lagen)".\parencite{Dos-lagen}
    Lagen ställer krav på offentliga aktörer så att de ska tillgänglighetsanpassa webbplatser och mobila application.
    Myndigheten för digital förvaltning (Digg) är ansvarig för att genomföra lagen och de utgår från WCAG standarden för att sätta krav på hemsidorna.

    \subsubsection{Offentlig aktör}
    Hur en offentlig aktör definieras finns förklarat i \textcite{Dos-lagen} men det kan enklare förklaras som offentlig information från staten som statliga och kommunala myndigheter samt sammanslutningar till dem.
    Saker som rör utbildning, skola, sjukvård ock omsorg måste också följa lagen.
    Även om det ägs privat så kommer de behöva följa kraven från Dos-lagen.

    Det kan enklare förklaras som statliga och kommunala myndigheter, beslutstagande församlingar i kommuner och regioner, skola och sjukvård.\parencite{Om_Dos-lage}


    \subsection{Verktyg}
    behöver formulera mitt spreadsheet för att veta vilka verktyg att prata om.

    \subsection{Komunala Hemsidor}
    För att meddela information om en kommun digital har kommuner hemsidor som det styr innehållet på och lägger upp information om kommunen.
    Det som är intressant om kommunal hemsidor är att alla är inte bara en hemsida som ser lika dan ut för varje kommun utan de ser annorlunda ut men vissa kommuner har hemsidor som ser likadana ut bara varierande innehåll.
    Skillnaden i hur kommunala hemsidor ser ut betyder att det finns en skillnad på hur de är uppbyggda och hur det är anpassade för användarvänlighet för webben.


    
    \section{Metod och Material}

    Undersökningen utgick på att testa hur om kommunernas hemsidor uppfyller alla krav i WCAG 2.2.
    Digg och W3C har givit ut en listor över hur kraven kan uppnås i lagen och WCAG 2.2.
    Undersökningen utgick ifrån en sammanställning över kraven och hur de kan mötas i en tabell (f).
    Utifrån den tabellen testades alla kraven på hemsidorna.
    Nedan fins en förklaring av vad varje colum i tabellen består av.
    \begin{enumerate}
        \item Själva WCAG 2.2 kravet som skulle testas.
        \item Hur WCAG 2.2 kravet ska testas
        \item På vilken hemsida som kravet testades
        \item Om hemsidan uppfyllde kravet
        \item Kommentarer om hemsidan
        \item 3-5 repeterade 2 gånger till för att få plats för 2 hemsidor till.
    \end{enumerate}

    (temp) google spreadsheet https://docs.google.com/spreadsheets/d/1SNX30NQTKxSrTisLqjCoPWqg7aKbZACcXsoCPYrQb6c/edit?usp=sharing
    \section{Resultat}
    
    \section{Diskusion}

    \section{Referenser}

    \printbibliography

\end{document}
